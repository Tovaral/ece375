\documentclass[12pt,letterpaper]{article}
\usepackage{anysize}
\marginsize{2cm}{2cm}{1cm}{1cm}

\begin{document}

\begin{titlepage}
    \vspace*{4cm}
    \begin{flushright}
    {\huge
        ECE 375 Lab 4\\[1cm]
    }
    {\large
        Introduction to AVR Tools
    }
    \end{flushright}
    \begin{flushleft}
    Lab Time: Monday Noon-2:00pm
    \end{flushleft}
    \begin{flushright}
    Ian Kronquist
    \vfill
    \rule{5in}{.5mm}\\
    TA Signature
    \end{flushright}

\end{titlepage}

\section{Prelab}
    turn right
The answers to the prelab questions can be found in the AVR Starters Guide and
the AVR Instruction Set.
\begin{enumerate}
    \item For this lab, you will be asked to perform arithmetic operations on numbers that
are larger than 8-bits. To do this, you should understand the different arithmetic operations supported by the AVR Architecture. List and describe all the different forms of ADD, SUB, and MUL (i.e. ADC, SUBI, MULF, etc.).\\
\begin{enumerate}
    \item $ADD$  Add the two registers, and put the result in $Rd$. Do not consult the carry bit.\\
    \item $ADC$  Add the two registers and the carry bit, and put the result in $Rd$.\\
    \item $ADIW$ Add a 6 bit constant to a register and carry into the next register. Only works on the upper 4 register pairs.\\
    \item $SUB$ Subtract the two registers without regard to the carry bit.\\
    \item $SUBI$ Subtract the value in the instruction from $Rd$.\\
    \item $SBC$ Subtract the two registers with regard to the carry bit.\\
    \item $SBCI$ Subtract the value in the instruction from the register with regard to the carry bit.\\
    \item $SBIW$ Subtract a 6 bit constant to a register and carry into the next register. Only works on the upper 4 register pairs.\\
    \item $MUL$ Perform unsigned multiplication of the two registers and put the result in registers $R1$ and $R0$.\\
    \item $MULS$ Perform two's complement signed multiplication of the two registers and put the result in registers $R1$ and $R0$.\\
    \item $MULSU$ Multiply a signed number with an unsigned one. The result will go into $R1$ and $R0$.\\
    \item $FMUL$ Same as $MUL$, but shifts the result one bit to the left. Useful for multiplying fixed point numbers.\\
    \item $FMULS$ Same as $MULS$, but shifts the result one bit to the left. Useful for multiplying fixed point numbers.\\
    \item $FMULSU$ Same as $MULSU$, but shifts the result on bit to the left. Useful for multiplying fixed point numbers.\\
\end{enumerate}


    \item Write pseudo-code that describes a function that will take two 16-bit numbers in data memory addresses \$0110-\$0111 and \$0121-\$0122 and add them together. The function will then store the resulting 16-bit number at the address \$0100-\$0101. (Hint: The upper address corresponds to the high byte of the number and don’t forget about the carry in bit.)\$
  \begin{verbatim}
    load r0, (0x0110)
    load r1, (0x0111)

    load r2, (0x0121)
    load r3, (0x0122)

    add r0, r2
    add_with_carry r1, r3

    store 0x0100, r0
    store 0x0101, r1
  \end{verbatim}
\end{enumerate}

\end{document}
