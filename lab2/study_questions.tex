% template created by: Russell Haering. arr. Joseph Crop
\documentclass[12pt,letterpaper]{article}
\usepackage{anysize}
\marginsize{2cm}{2cm}{1cm}{1cm}

\begin{document}

\begin{titlepage}
    \vspace*{4cm}
    \begin{flushright}
    {\huge
        ECE 375 Lab 2\\[1cm]
    }
    {\large
         C->Assembly->Machine Code->TekBot
    }
    \end{flushright}
    \begin{flushleft}
    Lab Time: Monday Noon-2:00pm
    \end{flushleft}
    \begin{flushright}
    Ian Kronquist
    \vfill
    \rule{5in}{.5mm}\\
    TA Signature
    \end{flushright}

\end{titlepage}

\section{Introduction}
Write a short summary that details what you did and why, explain any problems you may have encountered, and answer the questions below. Your write up and code must be submitted to your TA at the beginning of class the week following the lab. NO LATE WORK IS ACCEPTED.

For this lab we wrote a C program which turned the tekbot into a bumper car. When it would run into a wall it would reverse and turn away from the wall. We did this by polling the whiskers in a while loop. This is quite inefficient, but it is not timing critical and the processor would be in a busy wait loop anyway. The ports needed to be configured properly for input and output, and bit masks were required to determine whether the inputs had been hit. It was somewhat deceiving that the pins are active low, so the value needed to be reversed before applying the bit mask.

\section{Additional Questions}
\begin{enumerate}
    \item This lab required you to begin using new tools for compiling and
    downloading code to your AVR-enabled TekBot using the C language. Explain
    why it is beneficial to write code in a language that can be ‘cross
    compiled.’ Also explain some of the problems of writing in this way.\\

    Writing code in a machine independent language is a major advantage for
    programmers because they can write code once and have it run on many
    architectures. They no longer have to worry about many of the
    idiosyncrasies and nuances of a particular board. Code written in a higher
    level language is also easier to modify and maintain. One disadvantage is
    that these abstractions are not completely cost-free. For instance, with
    every function call in C there is a small amount of overhead which can
    sometimes be elided with handcrafted artisanal assembly.

    \item Your program does essentially the same thing as the assembly program
    you downloaded for Lab 1. Compare the size of the output hex files and
    explain the differences in size between them.\\

    The hex file form lab 1 was 472 bytes long. The hex file from lab 2 was 893
    bytes long. This is a 189\% increase. Part of this can be explained by the
    fact that the C program was a debug build. The C program also goes to
    considerable length to set up and maintain the program's stack.
    Additionally, not all optimizations were turned on. After inspecting the
    disassembled elf binary produced by the C compiler it looks as though two
    functions I thought would be inlined were not. The C program also has a
    number of fun interrupt handling functions.

\end{enumerate}

\section{Difficulties}
Some of the instructions were difficult to find or out of date. It was
unfortunate that this lab required us to use Atmel studio so it could not be
performed on a Mac or Linux system.

\section{Conclusion}
Embedded C affords a lot of flexibility and higher level descriptions of
program control flow. C compilers may not be as good at optimizing programs as
I thought they might.


\section{Source Code}
\begin{verbatim}
/*
    Challenge:
    Modify your TekBot so it can move objects across a tabletop. Your TekBot
    needs to push objects that it touches a short distance. 
*/

#define F_CPU 16000000
#include <avr/io.h>
#include <util/delay.h>
#include <stdio.h>

int main(void)
{
    // data direction register. 1 is output. 0 is input.
    DDRB =  0b11110000;    // Setup Port B for Input/Output
    DDRD =  0;             // Setup Port D for Input/Output
    PORTD = 0b11111111;    // Activate pull up resistor
    PORTB = 0b11110000;    // Turn off both motors


    PORTB = 0b01100000;    // Make TekBot move forward
    while (1) // Loop Forever
    {
        // a. TekBot hits object
        if (left_whisker_hit()) {
            // b. TekBot continues forwards for a short period of time
            _delay_ms(500);
            // c. TekBot backs up slightly
            PORTB = 0b00000000;
            // d. TekBot turns slightly towards the object
            PORTB = 0b00100000;    // Turn Left
            _delay_ms(1000);
            PORTB = 0b01100000;    // Make TekBot move forward
        } else if (right_whisker_hit()) {
            // b. TekBot continues forwards for a short period of time
            _delay_ms(500);
            // c. TekBot backs up slightly
            PORTB = 0b00000000;
            // d. TekBot turns slightly towards the object
            PORTB = 0b01000000;    // Turn Right
            _delay_ms(1000);
            PORTB = 0b01100000;    // Make TekBot move forward
        }
    }
}

int left_whisker_hit() {
    return ~PIND & 2;
}

int right_whisker_hit() {
    return ~PIND & 1;
}
\end{verbatim}
\end{document}
