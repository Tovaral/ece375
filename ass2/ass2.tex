% template created by: Russell Haering. arr. Joseph Crop
\documentclass[12pt,letterpaper]{article}
\usepackage{enumerate}
\usepackage{anysize}
\marginsize{2cm}{2cm}{1cm}{1cm}

\begin{document}

\begin{flushright}
{\large
ECE 375 Assignment 2\\
Ian Kronquist
}
\end{flushright}

\bigskip

\begin{enumerate}
    \item Assume that the Control Unit can dispatch the proper register Rd
    based on the instruction.\\
    Fetch Cycle
    \begin{enumerate}[i]
        \item MAR $\leftarrow$ PC  \emph{; Prepare address to fetch from memory}
        \item MDR $\leftarrow$ M(MAR), PC $\leftarrow$ PC + 1  \emph{; Fetch from memory and increment PC}
        \item IR $\leftarrow$ MDR$_{opcode}$, \emph{; Put opcode into IR to be read by CU}
        \item MAR $\leftarrow$ PC \emph{; Based on the opcode, continue the fetch cycle}
        \item MDR $\leftarrow$ M(MAR), PC $\leftarrow$ PC + 1 \emph{; Fetch the immediate value}
    \end{enumerate}

    Execute Cycle
    \begin{enumerate}[i]
        \item MAR $\leftarrow$ AC \emph{; Preserve the AC using MAR as a temp register}
        \item AC $\leftarrow$ Rd \emph{; Put the lower value of the register in mar}
        \item AC $\leftarrow$ AC + MDR, Rd $\leftarrow$ AC \emph{; Add MDR to AC, and open AC to data bus so the result will end up in the register}
        \item AC  $\leftarrow$ (Rd+1) \emph{; Move the upper value of the register to the AC}
        \item AC $\leftarrow$ AC + C, (Rd+1) $\leftarrow$ AC \emph{; Add carry bit to the upper value and open the data bus so the result will end up in the register}
        \item AC $\leftarrow$ MAR \emph{; Restore accumulator}
    \end{enumerate}
    \item Assume program memory is 16 bits wide, since it is not specified in the problem. Assume that we can use the MDR to write to the top or the bottom half of each byte in memory (this should be some simple control line which comes from the CU). Assume that the CALL instruction has the following layout:\\
    \begin{tabular}{l}
        16 Opcode Bits\\
        \hline
        16 Jump Address Bits
    \end{tabular}\\
    Fetch Cycle
    \begin{enumerate}[i]
        \item MAR $\leftarrow$ PC \emph{; Prepare address to fetch from memory}
        \item MDR $\leftarrow$ M(MAR)$_{upper}$, PC $\leftarrow$ PC + 1 \emph{; Fetch value from memory and increment PC to point to the jump address.}
        \item IR$_{upper}$ $\leftarrow$ MDR \emph{; Move the top part of the opcode to the IR}
        \item MDR $\leftarrow$ M(MAR)$_{lower}$ \emph{; Fetch the bottom part of the opcode}
        \item IR$_{lower}$ $\leftarrow$ MDR \emph{; Move the bottom part of the opcode to the IR}
        \item MAR $\leftarrow$ PC \emph{; Move the program counter to the MAR to fetch the jump address}
        \item MDR $\leftarrow$ M(MAR)$_{upper}$, PC $\leftarrow$ PC + 1 \emph{; Fetch the top half of the address. Increment PC to point past instruction.}
        \item TEMP$_{upper}$ $\leftarrow$ MDR \emph{; Move the top half of the address to the TEMP register}
        \item MDR $\leftarrow$ M(MAR)$_{lower}$ \emph{; Fetch the bottom half of the jump address}
        \item TEMP$_{lower}$ $\leftarrow$ MDR \emph{; Move the bottom half of the address to the TEMP register}
    \end{enumerate}

    Execute Cycle
    \begin{enumerate}[i]
        \item MAR $\leftarrow$ SP \emph{; Move the stack pointer to the MAR to prepare to save the current location of the PC to the stack1}
        \item MDR $\leftarrow$ PC$_{lower}$, SP $\leftarrow$ SP - 1 \emph{; Move the bottom half of the PC to the MDR and decrement the stack pointer to point past the new top of the stack.1}
        \item M(MAR)$_{lower}$ $\leftarrow$ MDR \emph{; Write the bottom half of the PC to memory1}
        \item MDR $\leftarrow$ PC$_{upper}$ \emph{; Move the top half of the PC to the MDR1}
        \item M(MAR)$_{upper}$ $\leftarrow$ MDR \emph{; Move the top half of the PC to the top of the stack.1}
        \item PC $\leftarrow$ TEMP \emph{; Move the jump address to the PC1}
    \end{enumerate}
    \item The loop will execute 4 times. At the end of the program the address
    CTR will contain the value 4.

    \begin{enumerate}[i]
    \item Set the origin of the program to 0x0. Immediately jump to the START
    label at 0x46
    \item Load the address of CTR into the X register using to load immediate
    instructions.
    \item Load the value 0xf0 into the 31st register. The binary representation
    of this value is 0b11110000.
    \item Set the fifth register to 0.
    \item Enter the main loop. Clear the carry bit.
    \item Rotate the 31st register left. The leftmost bit is placed in the
    carry register.
    \item If the carry bit is zero, jump to the SKIP label. Otherwise,
    increment the R5 register.
    \item At the SKIP label, compare R31 to the literal value 0x0. If they are
    the same, the Z bit is set and jump to the LOOP label.
    \item When the loop is finished, store the value of R5 in the CTR location.
    \item Enter an infinite loop. Finish the program.

    \end{enumerate}

    \item This is a table of the binary representation of code in memory\\
    \begin{tabular}{l l}
        Address & Binary Representation \\
        \hline
        0x00: & 1100, 0000, 0100, 0110 \\
        0x46: & 1110, 0000, 1011, 0001 \\
        0x47: & 1110, 0000, 1010, 0000 \\
        0x48: & 1110, 1111, 1111, 0000 \\
        0x49: & 0010, 0100, 0000, 0101 \\
        0x4a: & 1001, 0100, 1000, 1000 \\
        0x4b: & 0001, 1100, 0001, 1111 \\
        0x4c: & 1111, 0110, 0111, 0000 \\
        0x4d: & 1001, 0100, 0101, 0011 \\
        0x4e: & 0011, 0000, 1111, 0000 \\
        0x4f: & 1111, 0110, 0101, 0001 \\
        0x50: & 1001, 0010, 0101, 1100 \\
        0x51: & 1001, 0100, 0000, 1100 \\
        0x52: & 0000, 0000, 0101, 0001 \\
    \end{tabular}
\end{enumerate}
\end{document}
