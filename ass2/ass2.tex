% template created by: Russell Haering. arr. Joseph Crop
\documentclass[12pt,letterpaper]{article}
\usepackage{enumerate}
\usepackage{anysize}
\marginsize{2cm}{2cm}{1cm}{1cm}

\begin{document}

\begin{flushright}
{\large
ECE 375 Assignment 2\\
Ian Kronquist
}
\end{flushright}

\bigskip

\begin{enumerate}
    \item Assume that the Control Unit can dispatch the proper register Rd
    based on the instruction.\\
    Fetch Cycle
    \begin{enumerate}[i]
        \item MAR $\leftarrow$ PC
        \item MDR $\leftarrow$ M(MAR), PC $\leftarrow$ PC + 1
        \item IR $\leftarrow$ MDR$_{opcode}$, MAR $\leftarrow$ MDR$_{address}$
        \item MAR $\leftarrow$ PC
        \item MDR $\leftarrow$ M(MAR), PC $\leftarrow$ PC + 1
    \end{enumerate}

    Execute Cycle
    \begin{enumerate}[i]
        \item MAR $\leftarrow$ AC
        \item AC $\leftarrow$ Rd
        \item AC $\leftarrow$ AC + MDR, Rd $\leftarrow$ AC
        \item AC  $\leftarrow$ (Rd+1)
        \item AC $\leftarrow$ AC + C, (Rd+1) $\leftarrow$ AC
        \item AC $\leftarrow$ MAR
    \end{enumerate}
    \item Assume that the instruction has the following layout.\\
    \begin{tabular}{l}
        16 Opcode Bits\\
        \hline
        16 Least Significant Bits
    \end{tabular}\\
    Fetch Cycle
    \begin{enumerate}[i]
        \item MAR $\leftarrow$ PC
        \item MDR $\leftarrow$ M(MAR)$_{upper}$, PC $\leftarrow$ PC + 1
        \item IR$_{upper}$ $\leftarrow$ MDR
        \item MDR $\leftarrow$ M(MAR)$_{lower}$
        \item IR$_{lower}$ $\leftarrow$ MDR
        \item MAR $\leftarrow$ PC
        \item MDR $\leftarrow$ M(MAR)$_{lower}$, PC $\leftarrow$ PC + 1
        \item TEMP$_{upper}$ $\leftarrow$ MDR
        \item MDR $\leftarrow$ M(MAR)$_{lower}$
        \item TEMP$_{lower}$ $\leftarrow$ MDR
    \end{enumerate}

    Execute Cycle
    \begin{enumerate}[i]
        \item MAR $\leftarrow$ SP
        \item MDR $\leftarrow$ PC$_{lower}$, SP $\leftarrow$ SP - 1
        \item M(MAR) $\leftarrow$ MDR
        \item MAR $\leftarrow$ SP
        \item MDR $\leftarrow$ PC$_{upper}$
        \item M(MAR) $\leftarrow$ MDR
        \item PC $\leftarrow$ TEMP, SP $\leftarrow$ SP - 1
    \end{enumerate}
    \item The loop will execute 4 times. At the end of the program the address
    CTR will contain the value 4.

    \begin{enumerate}[i]
    \item Set the origin of the program to 0x0. Immediately jump to the START
    label at 0x46
    \item Load the address of CTR into the X register using to load immediate
    instructions.
    \item Load the value 0xf0 into the 31st register. The binary representation
    of this value is 0b11110000.
    \item Set the fifth register to 0.
    \item Enter the main loop. Clear the carry bit.
    \item Rotate the 31st register left. The leftmost bit is placed in the
    carry register.
    \item If the carry bit is zero, jump to the SKIP label. Otherwise,
    increment the R5 register.
    \item At the SKIP label, compare R31 to the literal value 0x0. If they are
    the same, the Z bit is set and jump to the LOOP label.
    \item When the loop is finished, store the value of R5 in the CTR location.
    \item Enter an infinite loop. Finish the program.

    \end{enumerate}

    \item This is a table of the binary representation of code in memory\\
    \begin{tabular}{l l}
        Address & Binary Representation \\
        \hline
        0x00: & 1100, 0000, 0100, 0110 \\
        0x46: & 1110, 0000, 1011, 0001 \\
        0x47: & 1110, 0000, 1010, 0000 \\
        0x48: & 1110, 1111, 1111, 0000 \\
        0x49: & 0010, 0100, 0000, 0101 \\
        0x4a: & 1001, 0100, 1000, 1000 \\
        0x4b: & 0001, 1100, 0001, 1111 \\
        0x4c: & 1111, 0110, 0111, 0000 \\
        0x4d: & 1001, 0100, 0101, 0011 \\
        0x4e: & 0011, 0000, 1111, 0000 \\
        0x4f: & 1111, 0110, 0101, 0001 \\
        0x50: & 1001, 0010, 0101, 1100 \\
        0x51: & 1001, 0100, 0000, 1100 \\
        0x52: & 0000, 0000, 0101, 0001 \\
    \end{tabular}
\end{enumerate}
\end{document}
