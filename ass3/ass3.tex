% template created by: Russell Haering. arr. Joseph Crop
\documentclass[12pt,letterpaper]{article}
\usepackage{enumerate}
\usepackage{anysize}
\marginsize{2cm}{2cm}{1cm}{1cm}

\begin{document}

\begin{flushright}
{\large
ECE 375 Assignment 3\\
Ian Kronquist
}
\end{flushright}

\bigskip

\begin{enumerate}
    \item
    \begin{enumerate}
        \item
        \begin{tabular}{l l l}
             Interrupt & Port & IVT Address \\
             \hline
             INT0 & PD0 & 0x02 \\
             INT3 & PD3 & 0x08 \\
             INT5 & PD5 & 0x0c \\
        \end{tabular}
        \item Interrupt 3 will trigger on a falling edge.\\
        \item PORTE and PORTD \\
        \item \begin{verbatim}
            ...
            ldi mpr,   $00
            out DDRD,  mpr
            out DDRE,  mpr
            ldi mpr,   0b00001000
            out PORTD, mpr
            ...
        \end{verbatim}
        \item \begin{verbatim}
            ISR_DevA:
                ldi mpr, 0x00
                out EIMSK, mpr
                ...
                ldi, mpr 0x2a
                out EIMSK, mpr
                ret
        \end{verbatim}
        \item The ISR\_DevA routine will be executed first.\\
    \end{enumerate}
    \item Assume that we can destroy the contents of mpr.
    \begin{verbatim}
        (1) ldi XL, low(BASEADDR)
        (2) ldi XH, high(BASEADDR)
        (3) in  mpr, PIND
        (4) add XL, mpr
        (5) adc XH, zero
        (6) ld  mpr, X
        (7) out PORTB, mpr
        (8) ret
    \end{verbatim}
    \item \begin{verbatim}
        .equ  prescalar 3035
        INIT:
            ; Initialize stack pointer to RAMEND
            ldi mpr, low(RAMEND)
            out SPL, mpr
            ldi mpr, high(RAMEND)
            out SPH, mpr

            ; Clear TCCR1A and TCCR1C.
            ; TCCR1A is mostly related to PWM
            ; and TCCR1C is mostl concerned with fast comparison.
            clr mpr
            out TCCR1A, mpr
            out TCCR1C, mpr

            ; Set the prescalar to 256
            ldi mpr, 0x04
            out TCCR1B, mpr

        WAIT:
            ; Preserve mpr.
            push mpr

            ; Load the prescalar.
            ; Remember to push the high byte first!
            ldi mpr, high(prescalar)
            sts TCNT1H, mpr
            ldi mpr, low(prescalar)
            sts TCNT1L, mpr

            ; Clear the timer overflow bit for timer 1.
            cbi TIFR, TOV1

            ; Enter the wait loop.
            ; Continuously poll TOV1 to see if it's been flipped
            LOOP:
                in   mpr, TIFR1
                ; Sets Z flag!
                andi mpr, (1<<TOV1)
                brne LOOP

            ; Cleanup
            pop mpr
            ret
    \end{verbatim}
    \item $UBRR = 16 Mhz/(8 \times 96k) -1 = 103$.
    \begin{verbatim}
        ; Initialize stack pointer
        ( 1) ldi mpr, high(RAMEND)
        ( 2) out SPL, mpr
        ( 3) ldi mpr, low(RAMEND)
        ( 4) out SPH, mpr

        ; Set up PIND for input. We can do this in one step!
        ( 5) cbi DDRD, 0x2
        ( 6) nop
        ; Alternatively do:
        ( 5) ldi mpr, (1<<PD2)
        ( 6) out DDRD, mpr

        ; Set the baud rate and enable double data rate
        ( 7) ldi mpr, 103
        ( 8) st UBRR1L, mpr
        ( 9) ldi, mpr, (1<<U2X1)
        (10) sts UCSR1A, mpr

        ; Enable receive interrupt
        (11) ldi mpr, (1<<RXCIE1)|(1<<RXEN1)
        (12) out UCSR1B, mpr

        ; Set frame format
        (13) ldi mpr, (1<<UCCSZ10)|(1<<UCCSZ11)|(1<<USBS1)|(1<<USBM1)
        (14) out UCSR1C, mpr

        ; Inside the interrupt handler
        (15) ldi mpr, UDR1
        (16) st  X+, mpr
    \end{verbatim}
\end{enumerate}
\end{document}
